\chapter{规范参考文献}
\paragraph{}
文中所提及以下文件,部分或全部内容构成本文档中的要求。凡是注日期的引用文件,仅引
用的版本适用。凡是不注日期的引用文件,其最新版本(包括所有的修改单)适用于本标准。
\begin{itemize}
  \item Ecma International, \textit{ECMAScript Language Specification}, Standard
    Ecma-262, third edition, 1999.
  \item ISO/IEC 2382 (all parts), \textit{Information technology - Vocabulary}
  \item ISO/IEC 9899:2011, \textit{Programming languages - C}
  \item ISO/IEC 9945:2003, \textit{Information Technology - Portable Operatiing
    System Interface (POSIX)}
  \item ISO/IEC 10646-1:1993, \textit{Information technology - Universal
      Multiple-Octet Coded Character Set (UCS) - Part 1: Architecture and Basic
    Multilingual Plane}
  \item ISO/IEC/IEEE 60559:2011, \textit{Information technology - Microprocessor
    Systems - Floating-Point arithmetic}
  \item ISO 80000-2:2009, \textit{Quantities and units - Part 2: Mathematical
    signs and symbols to be used in the natural sciences and technology}
\end{itemize}

\paragraph{}
ISO/IEC 9899:2011的第7章中描述的库在下文中称为C标准库\footnote{由第21章至第33章
和C.5中所述,C标准库是C++标准库的子集。}。

\paragraph{}
ISO/IEC 9945:2003中描述的操作系统接口在下文中称为POSIX。

\paragraph{}
标准Ecma-262中描述的ECMAScript语言规范在下文中称为ECMA-262。

