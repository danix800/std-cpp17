\chapter{一般原则}
\section{合规实现}

\paragraph{}
\textit{可诊断规则}集合包括本文档中的所有语法和语义规则,但那些包含“无需诊断”的明
确表示法或描述为导致“未定义行为”的规则除外。

\paragraph{}
虽然本文档仅说明了对C++实现的要求,但如果将这些要求表达为对程序,程序部分或程序
执行的要求,则这些要求通常更容易理解。这些要求具有以下含义:
\begin{enumerate}
  \item 如果一个程序没有违反本文档中的规则,则合规实现应在其资源限制内接受并正确
    执行\footnote{取决于所处理的数据,“正确执行”可能包括未定义行为;见第3章和
    第4.6节}该程序。
  \item 如果程序包含违反任何可诊断规则或出现本文档中描述为“有条件支持”的构造而实
    现不支持时,合规实现应至少给出一条诊断消息。
  \item 如果程序包含违反不需要诊断的规则,则本文档不对实现在该程序上作要求。
\end{enumerate}
「注:在模板参数推导和替换过程中,在其他上下文中需要诊断的某些构造会区别对待;见
第17.8.2节。」

\paragraph{}
对于类和类模板,库章节指定部分定义。 私有成员(第14章)不作规定,但每个实现应该
根据库章节的描述提供他们以完成定义。

\paragraph{}
对于函数,函数模板,对象和值,库章节指定声明。 实现应提供与库章节中描述一致的定义。

\paragraph{}
库中定义的名称具有命名空间作用域(10.3)。 C++翻译单元(5.2)通过包含适当的标准库
头(19.2)来获得对这些名称的访问。

\paragraph{}
库中的模板,类,函数和对象具有外部链接(6.5)。实现在必要时提供标准库实体定义,同
时组合翻译单元以形成完整的C++程序(5.2)。

\paragraph{}
定义两种实现:托管实现和独立实现。 对于托管实现,本文档定义了可用库集。 独立实现
是一种可以在没有操作系统的情况下执行的实现,并且具有包含某些语言支持库的实现定义
的库集(20.5.1.3)。

\paragraph{}
合规实现可能具有扩展(包括其他库函数),前提是它们不会改变任何格式良好程序的行为。
根据本文档,需要实现来诊断使用此类扩展的程序。但是,这样做之后,他们就可以编译和
执行这样的程序。

\paragraph{}
每个实现都应包含标识其不支持的所有条件支持构造的文档,并定义所有特定于语言环境的
特征。\footnote{本文档同时定义实现定义行为;见第4.6节。}

\section{本文档结构}
\paragraph{}
第15章至第19章描述C++语言。该描述包括第4.3节中所述形式的详细语法规范。为方便起见,
附件A重复了所有语法规范。

\paragraph{}
第21章至第33章和附件D(库章节)描述了C++标准库。 该描述包括以第20章中所述形式构成
库实体和宏的详细描述。 

\paragraph{}
附件B建议合规实现的能力下限。

\paragraph{}
附录C总结了C++自首次发布以来的演变,并详细解释了C++和C之间的差异。C++的某些特性
仅用于兼容性目的; 附件D描述了这些功能。

\paragraph{}
整个文档中,例子由“「例:”引入,由“」”结束。标注由“「注:”引入,由“」”结束。例和
注可以嵌套。

\section{语法记号}
\paragraph{}
本文档所用语法记号中,语法类别由斜体\textit{italic}表示,字面值和字符使用等宽
\texttt{width}字体表示。可选型在单独行列出,除少数情况下,一长串可选型集用
“one of”标记。如果可选型太长而无法放在一行上,则文本将从第一行缩进的后续行上继续。
可选的结束符或非结束符号由下标“opt”表示,如
\begin{lstlisting}[mathescape]
  { $expression_{opt}$ }
\end{lstlisting}
表示括号包含的可选表达式。

\paragraph{}
通常根据以下规则选择语法类别的名称:
\begin{enumerate}
  \item \textit{X-name}表示在确定其含义的上下文中使用标识符(例如
    \textit{class-name},\textit{typedef-name})。
  \item \textit{X-id}表示无上下文相关的标识符(例如\textit{qualified-id})。
  \item \textit{X-seq}表示无分隔的一个或多个\textit{X}(例如
    \textit{declaration-seq}是一个声明序列)。
  \item \textit{X-list}表示由逗号分隔的一个或多个\textit{X}(例如
    \textit{identifier-list}是一个逗号分隔的标识符列表)。
\end{enumerate}

\section{C++内存模型}
\paragraph{}
C++内存模型中的基本存储单元是\textit{字节}。一个字节至少足以包含基本执行字符集
(5.3)的任何成员和Unicode UTF-8编码形式的八位代码单元,并由连续的位序列组成,
其数量由实现定义\footnote{一个字节的位数由头<climits>中的宏CHAR\_BIT定义。}。最低
有效位称为\textit{低位};最高有效位称为\textit{高位}。C++程序可用的内存由一个或
多个连续字节序列组成。每个字节都有一个唯一的地址。

\paragraph{}
「注:类型表示在第6.9节描述。」

\paragraph{}
内存位置是标量类型的对象或者具有非零宽度的相邻位域的最大序列。「注:语言的各种功
能,例如引用和虚函数,可能涉及程序无法访问但由实现管理的附加内存位置。」两个或多
个执行线程(4.7)可以访问单独的内存位置而不会相互干扰。

\paragraph{}
「注:因此,位字段和相邻的非位字段在单独的内存位置中,因此可以由两个执行线程同时
更新而没有干扰。 这同样适用于两个位字段,如果一个在嵌套的结构中声明而另一个不是,
或者两者是由零长度位字段声明分隔,或者如果由非位字段声明分隔。如果它们之间的所有
字段都是非零宽度的位字段,则同时更新同一结构中的两个位字段是不安全的。」

\paragraph{}
「例:结构体声明
\begin{lstlisting}[mathescape]
  struct {
    char a;
    int  b : 5,
         c : 11,
           : 0,
         d : 8;
    struct {
      int ee : 8;
    } e;
  };
\end{lstlisting}
包含四个独立内存位置:成员a,位域d和e.ee每一个都是独立内存位置,可以同时修改而互
不干扰。位域b和c一起组成第四个内存位置。位域b和c不能同时修改,但b和a可以。」

\section{C++对象模型}

\paragraph{}
C++程序中的构造创建,销毁,引用,访问并操作对象。\textit{对象}由定义(6.1),
\textit{new表达式}(8.3.4),隐式切换联合活动成员(12.3)或创建临时对象时创建。
对象在其创建期(15.7),整个生存期(6.8)和销毁期(15.7)占据一段内存。「注:函
数不是对象,无论它是否和对象一样占据内存。」对象属性在创建时确定。对象可以有
名字(第6章)。对象有存储期(6.7),影响其生命期(6.8)。对象拥有类型(6.9)。某
些对象是多态的(13.3);实现生成关联于对象的信息,使其能够在程序执行期确定对象的
类型。对其他对象,其内所含值的解释由访问对象的\textit{表达式}(第8章)类型确定。

\paragraph{}
对象可以包含其他对象,称为\textit{子对象}。子对象可以是一个\textit{成员子对象}
(12.2),一个\textit{基类子对象}(第13章)或是一个数组元素。不是任何其他对象子
对象的对象称为\textit{完整对象}。如一个对象在成员子对象或数组元素\textit{e}(可
能在或不在其生命期内)所关联的存储内创建,且满足以下条件,那么所创建对象是包含
\textit{e}的对象的子对象:
\begin{enumerate}
  \item \textit{e}的包含对象生命期已开始且未结束,且
  \item 新对象的存储恰好覆盖\textit{e}所关联存储,且
  \item 新对象与\textit{e}的类型相同(忽略cv-限定)。
\end{enumerate}
「注:如子对象包含引用成员或\texttt{cont}子对象,则原子对象名不能用于访问新子对
象(6.8)。」 \\
「例:
\begin{lstlisting}
  struct X { const int n; };
  union U { X x; float f; };
  void tong() {
    U u = {{ 1 }};
    u.f = 5.f;                          // OK, creates new subobject of u (12.3)
    X *p = new (&u.x) X{2};             // OK, creates new subobject of u
    assert(p->n == 2);                  // OK
    assert(*std::launder(&u.x.n) == 2); // OK
    assert(u.x.n == 2); // undefined behavior, u.x does not name new subobject
  }
\end{lstlisting}
」

\paragraph{}
如一个完整对象在另一个类型为“大小为N的无符号字符数组”或“大小为N的std::byte数组”
(21.2.1)的对象\textit{e}所关联内存内创建(8.3.4),所创建对象由那些数组提供存
储,如果:
\begin{enumerate}
  \item \textit{e}的生命期已开始且未结束,且
  \item 新对象存储完全在\textit{e}内,且
  \item 没有更小的数组对象满足这些约束。
\end{enumerate}
「注:如数组的这一部分之前为其他对象提供存储,那个对象的生命期终止,因为其存储被
重用了。」 \\
「例:
\begin{lstlisting}
  template<typename ...T>
  struct AlignedUnion {
    alignas(T...) unsigned char data[max(sizeof(T)...)];
  };
  int f() {
    AlignedUnion<int, char> au;
    int *p = new (au.data) int;     // OK, au.data provides storage
    char *c = new (au.data) char(); // OK, ends lifetime of *p
    char *d = new (au.data + 1) char();
    return *c + *d; // OK
  }

  struct A { unsigned char a[32]; };
  struct B { unsigned char b[16]; };
  A a;
  B *b = new (a.a + 8) B; // a.a provides storage for *b
  int *p = new (b->b + 4) int; // b->b provides storage for *p
                               // a.a does not provide storage for *p (directly),
                               // but *p is nested within a (see below)
\end{lstlisting}
」

\paragraph{}
对象\textit{a}嵌套于对象\textit{b}内,如果:
\begin{enumerate}
  \item \textit{a}是\textit{b}的子对象,或
  \item \textit{b}为\textit{a}提供存储,或
  \item 存在对象\textit{c},\textit{a}嵌套于\textit{c}内,\textit{c}嵌套于
    \textit{b}内。
\end{enumerate}

\paragraph{}
对每一个对象\texttt{x},存在对象称为\texttt{x}的\textit{完整对象},由以下确定:
\begin{enumerate}
  \item 如\texttt{x}是完整对象,则\texttt{x}的完整对象为其自身。
  \item 否则,\texttt{x}的完整对象为包含\texttt{x}的(唯一)对象的完整对象。
\end{enumerate}

\paragraph{}
如一个完整对象,数据成员(12.2)或数组元素为类类型,则其类型被视为
\textit{最终派生类},以区别于任何基类子对象的类类型;最终派生类类型或非类类型对
象称为最终派生对象。

\paragraph{}
除非它是位字段(12.2.4),否则最终派生对象应具有非零大小并且应占用一个或多个存储
字节。基类子对象可以具有零大小。 平凡可复制或标准布局类型(6.9)对象应占用连续的
存储字节。

\paragraph{}
除非对象是零字段或零大小的基类子对象,否则该对象的地址是它占用的第一个字节的地
址。不是位字段的具有重叠生命周期的两个对象\textit{a}和\textit{b},如果一个嵌套在
另一个中,或者如果至少一个是零大小的基类子对象并且它们是不同类型,则可以具有相同
的地址; 否则有不同的地址。\footnote{根据“as-if”规则,如果程序无法观察到差异,则
允许实现在同一机器地址存储两个对象,或者根本不存在对象(4.6)。}
「例:
\begin{lstlisting}
  static const char test1 = 'x';
  static const char test2 = 'x';
  const bool b = &test1 != &test2; // always true
\end{lstlisting}
」

\paragraph{}
「注:C++提供了各种基本类型以及从现有类型组成新类型的几种方法(6.9)。」

\section{程序执行}

\paragraph{}
本文档中的语义描述定义了参数化的非确定性抽象机器。本文档不要求合规实现的结构。特
别是,它们不需要复制或模拟抽象机器的结构。 相反,需要合规实现来模拟如下所述的(仅)
抽象机器的可观察行为。\footnote{本条款有时被称为“as-if”规则,因为只要可以从该程序
可观察的行为中确定结果好像符合要求,实现就可以自由地忽略本文件的任何要求。例如,
实际实现不需要求值表达式某部分,如果它可以推断出它的值没有被使用,并且没有产生影
响程序的可观察行为的副作用。}

\paragraph{}
抽象机的某些方面和操作在本文档中描述为实现定义的(例如,\texttt{sizeof(int)})。
这些构成了抽象机器的参数。每个实施应包括描述其在这些方面的特征和行为的文档。
\footnote{这些文档也包括条件支持的结构和特定语言环境行为。见第4.1节。}
此类文档应定义与该实现相对应的抽象机器的实例(以下称为”对应实例”)。

\paragraph{}
抽象机的某些其他方面和操作在本文档中描述为未指定(例如,如果分配函数未能分配内存
,是否求值\textit{new初始化}中的表达式(8.3.4))。在可能的情况下,本文档定义了
一组允许的行为。 这些定义了抽象机器的非确定性方面。 因此,抽象机器的实例可以对给
定程序和给定输入具有多于一个可能的执行。

\paragraph{}
某些其他操作在本文档中描述为未定义(例如,尝试修改\texttt{const}对象的效果)。
「注:本文档对包含未定义行为的程序的行为不作要求。」

\paragraph{}
执行良构程序的合规实现应该产生与具有相同程序和相同输入的抽象机相应实例的可能执行
之一相同的可观察行为。但是,如果任何此类执行包含未定义操作,则本文档不对使用该输
入执行该程序的实现作要求(甚至不考虑第一个未定义操作之前的操作)。

\paragraph{}
具有自动存储期(6.7.3)的每个对象实例与每次进入其所在块相关联。这样的对象存在并
且在块的执行期间及块被暂停时(通过调用函数或接收信号)保持其最后存储值。

\paragraph{}
合规实现的最小需求为:
\begin{enumerate}
  \item 通过易失性泛左值访问严格按抽象机规则进行求值。
  \item 程序终止时,写入文件的所有数据应与根据抽象语义产生的程序执行的可能结果之
    一相同。
  \item 交互设备的输入和输出动态应以这样一种方式进行,即在程序等待输入之前提示输
    出实际被传送。构成交互设备的是实现定义的。
\end{enumerate}
这些统称为程序的\textit{可观察行为}。
「注:每个实现可以定义抽象和实际语义之间更严格的对应关系。」

\paragraph{}
「注:只有在操作符确实是结合或可交换的情况下,操作符才能按照通常的数学规则重新组
合。\footnote{重载运算符从不假设是结合或可交换的。}例如,在以下片段中
\begin{lstlisting}
  int a, b;
  /* ... */
  a = a + 32760 + b + 5;
\end{lstlisting}
由于运算符的结合性和优先级,表达式语句的行为与
\begin{lstlisting}
  a = (((a + 32760) + b) + 5);
\end{lstlisting}
完全相同。因此,\texttt{(a + 32760)}求和结果加上\texttt{b},然后结果再加上5,产
生结果并赋给\texttt{a}。在一台溢出会产生异常,\texttt{int}可表示值范围为
\texttt{[-32768, +32767]}的机器上,实现不能将该表达式重写成
\begin{lstlisting}
  a = ((a + 32765) + b);
\end{lstlisting}
或
\begin{lstlisting}
  a = (a + (b + 32765));
\end{lstlisting}
因为\texttt{a}和\texttt{b}的值可能分别为4和-8或-17和12。但是,在溢出不会产生异常
并且溢出结果是可逆的机器上,上述表达式语句可以通过上述任何一种方式重写,因为会出
现相同的结果。
」

\paragraph{}
\textit{构成表达式}定义如下:
\begin{enumerate}
  \item 表达式本身的构成表达式是其本身。
  \item \textit{braced-init-list}或(可能含括号)\textit{expression-list}的构成
    表达式为对应列表元素的构成表达式。
  \item 形如= \textit{initializer-clause}的\textit{brace-or-equal-initializer}的
    构成表达式为\textit{initializer-clause}的构成表达式。
\end{enumerate}
「例:
\begin{lstlisting}
  struct A { int x; }
  struct B { int y; struct A a; }
  B b = { 5, { 1 + 1 } };
\end{lstlisting}
用于初始化b的\textit{initializer}的构成表达式为5和1+1。」

\paragraph{}
表达式\texttt{e}的\textit{直接子表达式}为:
\begin{enumerate}
  \item \texttt{e}的操作数的构成表达式(第8章)。
  \item \texttt{e}所发起的任何函数调用。
  \item 如果\texttt{e}是\textit{lambda-expression}(8.1.5),由拷贝抓取的实体初
    始化,以及\textit{init-captures}的\textit{initializer}的构成表达式。
  \item 如果\texttt{e}是函数调用(8.2.2)或隐式调用一个函数,调用中的每个缺省参
    数的构成表达式,或
  \item 如果\texttt{e}创建聚合对象(11.6.1),其初始化中所用的每一个缺省成员初始
    化(12.2)的构成表达式。
\end{enumerate}

\paragraph{}
表达式\texttt{e}的\textit{子表达式}为\texttt{e}的直接子表达式或\texttt{e}的直接
子表达式的子表达式。「注:出现于\textit{lambda-expreession}的
\textit{compound-statement}里的表达式不是\textit{lambda-expression}的子表达式。
」

\paragraph{}
\textit{全表达式}指
\begin{enumerate}
  \item 不求值的操作数(第8章),
  \item 常表达式(8.20),
  \item \textit{init-declaration}(第11章)或\textit{mem-initializer}(15.6.2)
    ,包括初始化的构成表达式,
  \item 非临时对象生命期结束时产生的析构函数调用(15.2),或
  \item 不是某表达式子表达式,且不是某全表达式部分的表达式。
\end{enumerate}
如果定义语言构造以产生函数隐式调用,则该构造的使用被认为是用于该定义的表达式。
应用于表达式结果的转换以满足表达式出现的语言构造的要求也被认为是完整表达式的一部
分。对于初始化,执行实体的初始化(包括计算聚合的缺省成员初始值)也被视为全表达式
的一部分。「例:
\begin{lstlisting}
  struct S {
    S(int i) : I(i) { } // full-expression is initialization of I
    int& v() { return I; }
    ~S() noexcept(false) { }
  private:
    int I;
  };

  S s1(1);             // full-expression is call of S::S(int)
  void f() {
    S s2 = 2;          // full-expression is call of S::S(int)
    if (S(3).v())      // full-expression includes lvalue-to-rvalue and
                       // int to bool conversions, performed before
                       // temporary is deleted at end of full-expression
    { }
    bool b = noexcept(S()); // exception specification of destructor of S
                               // considered for noexcept
    // full-exception is destruction of s2 at end of block
  }

  struct B {
    B(S = S(0);
  };
  B b[2] = { B(), B() }; // full-exception is the entire initialization
                           // including the destruction of temporaries
\end{lstlisting}
」

\paragraph{}
「注:全表达式求值可以包含词法上不属于该全表达式部分的子表达式求值。例如,缺省参
数求值所涉及的子表达式被认为是在调用函数时创建,而不是定义缺省参数的表达式。」

\paragraph{}
\texttt{volatile}泛左值(6.10)所指代对象的读取,修改对象,调用库I/O函数或调用进
行以上任意操作的函数均是\textit{副作用},即对执行环境状态的修改。表达式(或子表
达式)\textit{求值}一般包括值计算和产生副作用。当调用I/O库函数返回时或访问
volatile泛左值计算时认为副作用完成,即使由调用(如I/O本身)或volatile访问所蕴含
的外部行为尚未完成。

\paragraph{}
\textit{前序}是单线程(4.7)所执行求值间的一种非对称,传递,配对关系,引入求值间
的一种偏序关系。给定任意两个求值\textit{A}和\textit{B},如果\textit{A}前序于
\textit{B}(或等价的,\textit{B}\textit{后序}于\textit{A}),则\textit{A}的执行
应在\textit{B}的执行之前。如果\textit{A}不前序于\textit{B}且\textit{B}不前序于
\textit{A},那么\textit{A}和\textit{B}为无序。「注:无序计算执行可以重叠。」
当要么\textit{A}前序于\textit{B}或\textit{B}前序于\textit{A}但未指明哪一个成立时
则\textit{A}和\textit{B}的求值为\textit{不确定性有序}。「注:不确定性有序求值不
会重叠,但任一个都可能先执行。」表达式\textit{X}前序于表达式\textit{Y},如果关联
于表达式\textit{X}的每一个值计算和每一个副作用前序于关联于表达式\textit{Y}的每一
个值计算和每一个副作用。

\paragraph{}
全表达式所关联的每个值计算和副作用前序于下一个待计算全表达式所关联的每一个值计算
和副作用。\footnote{如第15.2节所述,全表达式计算之后会产生临时对象的一个或多个析
构函数调用序列,通常是临时对象构造顺序的反序。}

\paragraph{}
除非另有说明,单个运算符的操作数求值和单个表达式的子表达式求值是无序的。「注:在
程序执行过程内一个计算多次的表达式中,其子表达式的无序和不确定性有序计算在不同求
值中不需要产生一致性结果。」运算符操作数的值计算前序于运算符结果的值计算。如一个
内存位置(4.4)上的副作用相对于同一内存位置上的另一副作用或使用同一内存位置上的
任何对象值的计算是无序的,且是潜在并发的(4.7),那么该行为是未定义的。「注:下
一节对潜在并发计算作类似但更复杂的约束。」
「例:
\begin{lstlisting}
  void g(int i) {
    i = 7, i++, i++;     // i becomes 9

    i = i++ +1;          // the value of i is incremented
    i = i++ + i;         // the behavior is undefined
    i = i + 1;           // the value of i is incremented
  }
\end{lstlisting}
」

\paragraph{}
当调用函数时(无论是否内联),任何实参表达式或指代所调用函数的后缀表达式,所关联
的每一个值计算和副作用前序于所调用函数体内任一表达式或语句的执行。对函数调用
\textit{F},出现于\textit{F}内的每一个计算\textit{A}和不出现于\textit{F}内的每一
个值计算\textit{B},但在同一线程和同一信号处理器(如果有)内计算,要么\textit{A}
前序于\textit{B},要么\textit{B}前序于\textit{A}。\footnote{换句话说就是函数执行
不会彼此交织。}「注:如果\textit{A}和\textit{B}没有说明顺序则为不确定性有序。」
即使翻译单元里没有出现对应函数调用语法,在C++里的多个上下文中都会引起函数调用
计算。「例:\textit{new-expression}求值调用一个或多个分配和构造函数;见8.3.4。
另外,转换函数(15.3.2)可以在无函数调用语法出现的情况下中调用。」被调用函数执行
的有序约束(如上所述)是函数调用执行时的特性,不论调用函数的表达式语法形式如何。

\paragraph{}
如信号处理器因调用\texttt{std::raise}函数而执行,那么处理器的执行后序于
\texttt{std::raise}函数调用,前序于其返回。「注:当信号处理器因其他原因接收到信
号时,其执行通常无序于程序其他部分。」

\section{多线程执行与数据竞争}
\paragraph{}
一个\textit{执行线程}(\textit{thread of execution})(也称\textit{线程})指程序
内的单个控制流,包括一个特定顶层函数的调用,并递归包含后续由线程所执行的每一个函
数调用。「注:当一个线程创建另一个线程时,新线程顶层函数的初始调用由新线程而不是
创建线程执行。」程序中每一个线程可以潜在访问程序中的每一个对象和函数。
\footnote{自动与线程存储期对象(6.7)关联于一个特定线程,且仅能通过指针或引用被
不同线程间接访问(6.9.2)。} 在宿主式实现中,C++程序可以有多个线程并发运行。每一
个线程按以下所定义方式进行。整个程序的执行包括其所有线程的执行。「注:通常执行可
以看成其所有线程的交织进行。然后,例如某些原子操作允许如下所述的简单交织不一样的
执行行为。」在自由式实现中,程序是否可以有多于一个线程由实现定义。

\paragraph{}
对于不是由调用\texttt{std::raise}引起的信号处理程序,哪个执行线程包含该信号处理
程序调用未指定。

\subsection{数据竞争}
\paragraph{}
根据以下规则,某一个特定点上,线程\textit{T}可见的对象的值为对象的初始值,由
\textit{T}赋给该对象的值或由其他线程赋给该对象的值。「注:某些情况下可能存在未定
义行为。本节大部分出于支持具有明确且细化的可见性约束的原子操作的想法。但是也隐含
地支持更受限制的程序的简单视图。」

\paragraph{}
两个表达式求值\FIT{冲突},如果其中一个修改一个内存位置(4.4)而另一个读或修改该
相同的内存位置。

\paragraph{}
标准库定义一系列的原子操作(第32章)和互斥操作(第33章),特别地称这些操作为同步
操作。这些操作在进行对另一个线程可见的赋值中起特殊作用。一个或多个内存位置上的同
步操作指一个消费操作、获取操作、释放操作或获取和释放操作两者。没有关联内存位置的
同步操作是一个围栏,可以是获取围栏、释放围栏、也可以是获取和释放围栏。此外,存在
放宽的原子操作,指原子的读-修改-写操作,但不是同步操作,有其特殊特征。「注:比如
获取互斥的调用会在包含互斥的位置上进行获取操作。对应的,释放同一互斥的调用会在相
同的位置上进行释放操作。非形式的,对A执行释放操作,会强制其他内存位置上的先前的
副作用,变得对其他线程可见,这些线程随后对A执行消耗或获取操作。即使和同步操作一
样,``放宽''的原子操作却不是同步操作,它们不会造成数据竞争。」

\paragraph{}
特定原子对象\textit{M}的所有修改按特定的全序进行,称为\textit{M}的
\textit{修改序}(\textit{modification order})。「注:每一个原子对象存在不同修改
序。不要求这些可以组合成所有对象的单个全序。一般而言这是不可能的,因为不同线程可
能会以不一致的顺序观察到不同对象的修改。」

\paragraph{}
由原子对象\textit{M}上的释放操作\textit{A}引导的\textit{释放序列}
(\textit{release  sequence})是\textit{M}的修改顺序的副作用的最大连续子序列,其
中第一操作是\textit{A},并且每个后续操作
\begin{enumerate}
  \item{由执行\textit{A}操作的同一线程进行,或}
  \item{是一个原子读-修改-写操作。}
\end{enumerate}

\paragraph{}
某些库调用\textit{同步于}另外的线程进行的其他库调用。比如,一个原子存储-释放与
从该存储取值的加载-获取同步(32.4)。「注:除指明的情况外,如下所述,读取稍后的
值不一定确保可见性。 这样的要求有时会干扰有效的实现。」「注:同步操作规范定义了
何时读另一个所写的值。对原子对象,定义是清晰的。给定互斥上的所有操作按单个全序进
行。每一个互斥获取``读取''上一个互斥释放``所写的值''。」

\paragraph{}
求值\textit{A}\textit{带有}求值\textit{B}的\textit{依赖}
(\textit{carry a dependency} to),如果
\begin{enumerate}
  \item{\textit{A}的值用作\textit{B}的操作数,除非
      \begin{enumerate}
        \item{\textit{B}是\texttt{std::kill\_dependency}(32.4)的任何特例化的调
          用,或}
        \item{\textit{A}是内置逻辑\texttt{AND}(\&\&,见8.14)或逻辑\texttt{OR}
          (||,见8.15)运算符的左操作数,或}
        \item{\textit{A}是条件运算符(\texttt{?:},见8.16)的左操作数,或}
        \item{\textit{A}是内置逗号运算符(\texttt{,})(8.19)的左操作数;}
      \end{enumerate}
      或}
  \item{\textit{A}写入标量对象或位域\textit{M},\textit{B}从\textit{M}中读
    \textit{A}所写值,且\textit{A}前序于\textit{B},或}
  \item{存在求值\textit{X},\textit{A}带有\textit{X}的依赖,\textit{X}带有
    \textit{B}的依赖。}
\end{enumerate}
「注:``带有依赖''是``前序''的子集,且类似的是严格线程内的。」

\paragraph{}
求值\textit{A}\textit{依赖前序}(\textit{dependency-ordered before})于求值
\textit{B},如果
\begin{enumerate}
  \item{\textit{A}在原子对象\textit{M}进行释放操作,另一个线程中,\textit{B}在
    \textit{M}上进行消费操作并读取\textit{A}所引导的释放序列中的任何副作用所写的
    值,或}
  \item{对某些求值\textit{X},\textit{A}依赖前序于\textit{X},\textit{X}带有
    \textit{B}的依赖。}
\end{enumerate}
「注:``依赖前序''关系类似于``同步于''关系,但使用释放/消费代替释放/获取。」

\paragraph{}
求值\textit{A}\textit{线程间先发生于}(\textit{inter-thread happens before})
\textit{B},如果
\begin{enumerate}
  \item{\textit{A}同步于\textit{B},或}
  \item{\textit{A}依赖前序于\textit{B},或}
  \item{对某些求值\textit{X}
      \begin{enumerate}
        \item{\textit{A}同步于\textit{X}且\textit{X}前序于\textit{B},或}
        \item{\textit{A}前序于\textit{X}且\textit{X}线程间先发生于\textit{B},}
        \item{\textit{A}线程间先发生于\textit{X}且\textit{X}线程间先发生于
          \textit{B}。}
      \end{enumerate}
    }
\end{enumerate}
「注:``线程间先发生于''关系描述``前序''、``同步于''和``依赖前序''关系的任意
连接,除两个例外。第一个例外为连接不能以``依赖前序''加上``前序''结束。原因是
``依赖前序''关系中的一个消费操作仅对该消费操作带有依赖的那些操作提供有序性。该限
制只对其这种连接结尾适用的原因是任何后序的释放操作将会为先前的消费操作提供所需有
序性。第二个例外是连接不允许仅包含``前序''。原因是(1)允许``线程间先发生于''关
系是传递封闭的,(2)``先发生于''关系,以下将定义,提供仅包含``前序''的关系。」

\paragraph{}
求值\textit{A}\textit{先发生于}(\textit{happens before})求值\textit{B}(或等
价的,\textit{B}\textit{后发生于}\textit{A}),如果
\begin{enumerate}
  \item{\textit{A}前序于\textit{B},或}
  \item{\textit{A}线程间先发生于\textit{B}。}
\end{enumerate}
实现应该确保程序不会出现``先发生于''关系的环。「注:反之环应仅通过消费操作的
使用才可能。」

\paragraph{}
求值\textit{A}\textit{严格先发生于}求值\textit{B},如果以下任一成立
\begin{enumerate}
  \item{\textit{A}前序于\textit{B},或}
  \item{\textit{A}同步于\textit{B},或}
  \item{\textit{A}严格先发生于\textit{X}且\textit{X}严格先发生于\textit{B}。}
\end{enumerate}
「注:不存在消费操作时,先发生于和严格先发生于关系是等价的。严格先发生于本质上排
除了消费操作。」

\paragraph{}
标量对象或位域\textit{M}上相对于\textit{M}的值计算\textit{B}的
\textit{可见副作用}(\textit{visible side effect})\textit{A}满足条件:
\begin{enumerate}
  \item{\textit{A}先发生于\textit{B}且}
  \item{没有其他副作用\textit{X}于\textit{M}使得\textit{A}先发生于\textit{X}且
    \textit{X}先发生于\textit{B}。}
\end{enumerate}
由求值\textit{B}所确定,非原子的对象或位域\textit{M}的值,应该是可见副作用
\textit{A}所存的值。「注:如果关于哪个副作用于非原子的对象或位域是可见的存在歧义
则该行为要么未指定要么未定义。」「注:这说明了普通对象上的操作不是可见重排序的。
无数据竞争的话这实际上不可检测,但如同以下定义,连同对原子使用上的合适限制,这
对确保数据竞争对应于单个交织(序列一致性)执行中的数据竞争是必要的。」

\paragraph{}
由求值\textit{B}所确定,原子对象\textit{M}的值应该是修改\textit{M}的某些副作用
\textit{A}所存的值,这里\textit{B}不先发生于\textit{A}。「注:该副作用的集合也
受此处描述的剩下的规则,特别是以下的一致性要求所限制。」

\paragraph{}
如果修改原子对象\textit{M}的操作\textit{A}先发生于修改\textit{M}的操作\textit{B}
则在\textit{M}的修改序中\textit{A}应该早于\textit{B}。「注:该要求称为写-写一致
性。」

\paragraph{}
如果原子对象\textit{M}的值计算\textit{A}先发生于\textit{M}的值计算\textit{B},且
\textit{A}从\textit{M}的一个副作用\textit{X}取值,则\textit{B}所计算的值应该要么
是\textit{X}所存的值或\textit{M}上的副作用\textit{Y}所存的值,这里\textit{Y}在
\textit{M}的修改序中跟在\textit{X}的后面。「注:该要求称为读-读一致性。」

\paragraph{}
如果原子对象\textit{M}的值计算\textit{A}发生在修改\textit{M}的操作\textit{B}之
前,则\textit{A}应从\textit{M}的副作用\textit{X}中取其值,其中\textit{X}在
\textit{M}的修改序中位于\textit{B}之前。「注:该要求称为读-写一致性。」

\paragraph{}
如果原子对象\textit{M}上的副作用\textit{X}发生在\textit{M}的值计算\textit{B}之
前,那么求值\textit{B}应从\textit{X}或\textit{M}的修改序中\textit{X}之后的副作
用\textit{Y}中取其值。「注:该要求被称为写-读一致性。」

\paragraph{}
「注:前面的四个一致性要求实际上不允许编译器将原子操作重新排序到单个对象,即使这
两个操作都是放宽松的载入。这有效地保证了C++原子操作可用的大部分硬件所提供的缓存
一致性。」

\paragraph{}
「注:原子载入所观察到的值取决于``先发生于''关系,而这取决于原子载入所观察到的
值。预期的读是必须存在原子载入与他们观察到的修改的关联,连同适当选择的修改顺序
和如上所述的``先发生于''关系,满足此处要求的结果约束。」

\paragraph{}
两个操作是\textit{潜在并发}(\textit{potentially concurrent})的,如果
\begin{enumerate}
  \item{它们由不同线程执行,或}
  \item{它们无序,至少一个由信号处理程序执行,且不同时由同一个信号处理程序执行。}
\end{enumerate}
如果包括两个潜在并发的冲突操作,至少一个非原子,且均不先发生于彼此,除以下讨论的
信号处理程序的特殊情形外,则程序的执行包含\textit{数据竞争}
(\textit{data race})。任何这样数据竞争都会产生未定义行为。「注:可以证明,正确
使用互斥和\texttt{memory\_order\_seq\_cst}以防止所有数据竞争且不使用其他同步操作
的程序行为就好像其组成线程的执行操作只是交错的,对象的每个值计算都取自交错执行中
该对象的最后一个副作用。这通常称为``顺序一致性''。然而这仅对无数据竞争程序成立,
且无数据竞争程序不能观察到不改变单线程程序语义的大部分程序变换。事实上,大部分单
线程程序变换仍然允许,因为任何行为不同的程序必定进行了未定义操作。」

\paragraph{}
如果两个对\texttt{volatile std::sig\_atomic\_t}类型的同一对象访问出现在同一线程
中则不产生数据竞争,即使一个或多个出现在信号处理程序中。对每一个信号处理程序调用
由线程进行的调用信号处理程序的求值可以分成\textit{A}和\textit{B}两组,使得
\textit{B}中没有求值先发生于\中没有求值先发生于textit{A}中求值,且这种
\texttt{volatile std::sig\_atomic\_t}对象求值的值取自如同\textit{A}中求值先发生
于信号处理程序的执行并且信号处理程序的执行先发生于\textit{B}中所有求值。

\paragraph{}
「注:引入对不由抽象机修改的潜在共享内存赋值的编译器变换通常被本文档所排除,因这
种赋值在抽象机执行未遇到数据竞争的情况下可能覆盖不同线程中另一个赋值。包括覆盖不
同内存位置的相邻数据成员的赋值的实现。有问题的原子可能有别名的情况下原子载入的重
排序通常也被排除,因为这可能违反一致性规则。」

\paragraph{}
「注:引入潜在共享内存猜测性读的变换可能不能保留本文档中定义的C++程序语义,因为
他们潜在地引入了数据竞争。然而在数据竞争定义清晰的机器作为目标的优化编译器环境中
它们通常是有效的。对不允许竞争或不提供硬件竞争检测的假想机器它们是无效的。」

\subsection{继续执行(Forward progress)}
\paragraph{}
实现可能假定任何线程最终都会执行以下之一:
\begin{enumerate}
  \item{终止,}
  \item{调用库I/O函数,}
  \item{访问volatile泛左值,}
  \item{进行同步或原子操作。}
\end{enumerate}
「注:其目的是允许编译器变换,比如空循环消除,即使不能证明循环终止。」

\paragraph{}
定义为无锁(32.8)或指明无锁(32.5)的原子函数执行是\textit{无锁执行}。
\begin{enumerate}
  \item{如果标准库函数中仅有一个未被阻塞,则无锁执行应该完成。「注:并发执行线程
    可能阻止无锁执行进行下去。比如,该情形会在锁定载入(load-locked),条件存储
    (store-conditional)实现中出现。该属性有时称为无障碍执行。」}
  \item{当一个或多个无锁执行并发运行时,至少一个应该完成。「注:对某些实现来说提
    供该效果的绝对保证是困难的,因为来自其他线程不断的特别是不合时宜的干涉可能阻
    止线程继续执行,如在锁定载入和条件存储指令间因不相关的目的而不断偷取缓存线。
    实现应该确保在期望的操作条件下这种效果不会无限延迟程序执行,而这样的异常行为
    可以安全地被程序员所忽略。本文档外该属性有时称为无锁执行。」}
\end{enumerate}

\paragraph{}
在线程执行过程中,以下每一条被称作\textit{执行步骤}:
\begin{enumerate}
  \item{执行线程的终止,}
  \item{访问volatile泛左值,或}
  \item{调用库I/O函数、同步操作或原子操作的完成。}
\end{enumerate}

\paragraph{}
带阻塞(3.6)标准库函数调用在等待阻塞条件满足的过程中被认为是连续执行其执行步骤
的。「例:一个阻塞直到I/O操作完成的库I/O函数被认为是在连续地检测操作是否完成。每
一次这样的检测可能包括一个或多个执行步骤,比如使用抽象机的可观察行为。」

\paragraph{}
「注:由于这一点以及关于执行线程最终必须执行的前述要求,因此没有执行步骤发生,任
何执行线程都不能永久执行。」

\paragraph{}
当执行步骤发生或无锁执行未完成时,执行线程会进行,因为在标准库函数中没有其他被阻
塞的并发线程(见上文)。

\paragraph{}
对于提供\textit{并发进行保证}(\textit{concurrent forword progress guarantees})
的执行线程,实现应该确保线程最终会在未终止时继续进行。「注:无论其他执行线程
(如果有)是否已经或正在取得进展,这都是必需的。 最终满足此要求意味着这将在未指
定但有限的时间内发生。」

\paragraph{}
由实现创建用于执行\texttt{main}(6.6.1)的执行线程以及由\texttt{std::threads}
(33.3.2)所创建的执行线程是否提供并发进行保证是由实现定义的。「注:通用目的实现
应该提供这些保证。」

\paragraph{}
对于提供\textit{并行进行保证}(\textit{parallel forward progress guarantees})的
执行线程,如果其尚未执行任何执行步骤,实现不要求确保线程最终将取得进展;一旦线程
执行了一个步骤,实现应该提供并发进行保证。

\paragraph{}
「注:这并不指明要求何时开始线程执行,而这通常是由创建该执行线程的实体所指定的。
比如,提供并发进行保证且依次以任意顺序执行一组任务中的任务的执行线程对这些任务满
足并行进行保证要求。」

\paragraph{}
对提供\textit{弱并行进行保证}的执行线程,实现不确保线程最终取得进展。

\paragraph{}
「注:无论其他线程是否取得进展,不能期望提供弱并行进行保证的执行线程取得进展;然
而使用如下定义的进展保证代理的阻塞可用于确保此类执行线程最终取得进展。」

\paragraph{}
并发进行保证比并行进行保证更强,并行进行保证比弱并行进行保证更强。「注:比如,某
些执行线程间同步可能仅当执行线程提供并行进行保证时才能取得进展,但在弱并行进行保
证下却不能取得进展。」

\paragraph{}
当执行线程\textit{P}指定被一组执行线程\textit{S}的完成
\textit{使用进展保证代理阻塞},那么在\textit{P}被\textit{S}阻塞的整个时间段中,
实现应该确保\textit{S}的至少一个执行线程所提供的进展保证至少和\textit{P}的进展保
证一样强。「注:具体是\textit{S}中的哪个线程或有多少执行步骤未指定。该加强不是永
久性的,并且不一定适用于受影响的执行线程的剩余生命周期。只要\textit{P}被阻塞,实
现必须最终从\textit{S}中选择并潜在加强一个执行线程。」\textit{S}中执行线程一旦终
止即从\textit{S}中移除。一旦\textit{S}为空,则\textit{P}不再阻塞。

\paragraph{}
「注:因此,执行线程\textit{B}可以暂时提供一段时间有效更强的进展保证,这是由于第
二个执行线程\textit{A}被其使用进展保证代理阻塞。 反过来,如果\textit{B}随后被
\textit{C}进展保证代理阻止,这可能也会暂时为\textit{C}提供更强的进展保证。」

\paragraph{}
「注:如果\textit{S}中的所有执行线程都完成执行(例如,终止且没有错误地使用阻塞同
步),那么P使用进展保证代理的阻塞操作的执行将不会导致P的进展保证被有效削弱。」

\paragraph{}
「注:这并没有消除线程执行的关于提供并行或弱并行进展保证的阻塞同步约束,因为实现
不要求强化特定的执行线程,其过于薄弱的进展保证阻止了整体进度。」

\paragraph{}
实现应该确保原子或同步操作所赋的最后一个值(按修改顺序)在有限的时间内对所有其他
线程可见。

\section{致谢}
\paragraph{}
本文档所描述的C++编程语言基于Stroustrup: \textit{The C++ Programming Language}
(second edition, Addition-Wesley Publishing Company, ISBN 0-201-53992-6,
copyright \copyright 1991 AT\&T)中第R章(参考手册)中所描述的语言。该语言基于C编
程语言,在Kernighan and Ritchie: \textit{The C Programming Language}
(Prentice-Hall, 1978, ISBN 0-13-110163-3, copyright \copyright 1978 AT\&T)附录A
中所述。

\paragraph{}
本文档中标准库章节部分基于P.J. Plauger的工作,发表于\textit{The Draft Standard
C++ Library} (Prentice-Hall, ISBN 0-13-117003-1, copyright \copyright 1995 P.J.
Plauger)。

\paragraph{}
POSIX\textregistered 是Institute of Electrical and Electronic Engineers, Inc的注
册商标。

\paragraph{}
ECMAScript\textregistered 是Ecma International的注册商标。

\paragraph{}
保留这些原件的所有权利。
