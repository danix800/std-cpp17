\chapter*{前言}
\addcontentsline{toc}{chapter}{前言}
\markboth{Introduction}{}

ISO(国际标准化组织,the International Organization for Standization)是一个全球
性的国家标准联合会机构(ISO成员机构)。准备国际标准的工作通常通过ISO技术委员会进
行。对成立技术委员会的主体感兴趣的每个成员机构都有权在该委员会中有代表。 与ISO联
络的国际组织,政府和非政府组织也参与了这项工作。 ISO在电工技术标准化的所有方面与
国际电工委员会(IEC)密切合作。

描述了用于开发本文档的程序以及用于进一步维护的程序在ISO/IEC指令第1部分中。特别是
不同类型所需的不同批准标准应注意ISO文件。本文件是根据该文件的编辑规则起草的
ISO/IEC指令,第2部分
(见\href{https://www.iso.org/directives}{www.iso.org/directives})。

需要注意的是,本文件的某些要素可能是专利的主题权利。ISO不负责识别任何或所有此类
专利权。任何专利的细节在文件开发过程中确定的权利将在引言和/或ISO列表中收到的专利
声明(见\href{https://www.iso.org/patents}{www.iso.org/patents})。

本文档中使用的任何商标名称是为方便用户所提供的信息,而不是构成认可。

关于标准的自愿性质的解释,与合格评定相关的ISO特定术语和表达的含义,以及ISO在技术
性贸易壁垒(TBT)中遵守世界贸易组织(WTO)原则的信息,请参阅以下网址: \\
\href{www.iso.org/iso/foreword.html}{www.iso.org/iso/foreword.html}

...

第五版取消并取代了第四版(ISO/IEC 14882:2014),该版本已经过技术修订。

与上一版相比的主要变化如下:
\begin{itemize}
  \item 表达式求值顺序作了更多的规定
  \item 去除三字母词
  \item 强制拷贝消除导致的值范畴调整
  \item ...
\end{itemize}

